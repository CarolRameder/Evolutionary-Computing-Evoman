\section{Related Work}

Self-adaption is defined according to Eiben \cite{series/ncs/EibenS03} as the dynamic changes of parameters during evolution based on the inclusion of parameters within an genetic encoding that is exposed itself to evolutionary process and pressures. Teo \cite{Teo2006SelfadaptiveMF} reported that their self-adaptive Gaussian-based mutation method dramatically outperformed a standard algorithm with a standard Gaussian mutation method. Accordingly, the self-adaptive mutation enabled the algorithm to escape local optima in a highly deceptive fitness landscape and hence successfully located global optimal solutions. To extend this finding to a different environmental set-up, this research aims to investigate the comparison between the self-adaptive mutation method and a standard Gaussian mutation method in the environment of the EvoMan framework \cite{Arajo2016EvolvingAG}.


