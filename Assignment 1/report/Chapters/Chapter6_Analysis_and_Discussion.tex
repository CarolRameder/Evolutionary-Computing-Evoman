\section{Analysis and Discussion}

Based on the experimental results, the average fitness value of both algorithms rises rapidly during the first five generations when facing any of the enemies, indicating that the evolution process works successfully.\par

Regarding the fitness performance, the main difference between the two EAs is the average fitness obtained over the generations. EA1 achieves higher fitness on average than EA2 regardless of the chosen enemy. Additionally, while testing the best individual of each training run, the average individual gain for the EA1 is consistently high and there is no large variation, whilst the EA2 fluctuates between a high range of different values. This means that the best individual of EA2 is not performing coherently in every experiment. These surprising findings emphasize that EA2 is not performing superior in comparison to EA1.\par

Furthermore, the plots show that high maximum fitness values are reached even from the first generation and remain constant throughout the evolution process. High maximum values in early generations could possibly be explained by a low performance of the enemy or the high population size that is used.\par

Moreover, the standard deviation of both EA fitness values is considerably higher for the Quickman enemy comparing to the other two enemies. A possible reason for this could be the structure of the environment for each game, considering that the Airman and Metalman enemy have a plain ground on their map, while the ground of Quickman yields some obstacles. These obstacles bring diversity that could slow down the evolution process and potentially cause a high difference between highly fit individuals and weaker ones. The similar development of both algorithms is an argument that the environment causes the high variations and not the differing mutation methods.\par

One possible reason for the weaker results of EA2 might be a missing survival selection. Since the self-adaptive mutation relies on the combination of all sigma values of an individual to achieve the best results, a multi-parent crossover might lead to a disconnection of coherent sigmas. Moreover, since the parents are completely replaced with the offspring, groups of well-functioning sigmas might get destroyed in the process of recombination and lost in the genetic pool. Therefore, for future research, it might be interesting to investigate how a different crossover method and an inclusion of survival selection effects the comparison between EAs using self-adaptive mutation and Gaussian mutation.\par


% For both algorithms, the average fitness value rises rapidly during the first five generations when facing any of the enemies, thus the evolution process works successfully.
% Furthermore, the plots show that high maximum fitness values are reached even from the first generation and remain constant throughout the evolution process. High maximum values in early generations could possibly be explained by a low performance of the enemy or the high population size that is used.

% Regarding the fitness performance, the main difference between the two EAs is the average fitness obtained over the generations. EA1 achieves higher fitness on average than EA2 regardless of the chosen enemy. Additionally, while testing the best individual of each training run, the average individual gain for the EA1 is consistently high and there is no large variation, whilst the EA2 fluctuates between a high range of different values. This means that the best individual of EA2 is not performing coherently in every experiment. These surprising findings emphasize that EA2 is not performing superior in comparison to EA1.
