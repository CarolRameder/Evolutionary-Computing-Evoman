\section{Introduction}
% whether an EA witha self-adaptive mutation method can be seen as superior in termsof fitness performance in comparison to a standard EA using astandard Gaussian mutation.

Evolutionary algorithms (EAs) are optimization algorithms inspired by the biological evolution. With the concept of the biological processes of selection, recombination and mutation of a population over generations, EAs are able to evolve in a complex environment based on a fitness function. 

In our last paper, we investigated whether a dynamically adaptive diversity for individuals, implemented with the self-adaptive mutation method, is a better optimization method for training an agent in the EvoMan framework \cite{defranca2020evoman} than an EA with a standard Gaussian mutation method. Our results did not indicate that the self-adaptive mutation is superior. As a possible reason, we suggested that a different survivor selection strategy might enhance the positive implications the self-adaptive mutation method has to offer. Therefore, in the current paper, we aimed to take up our previous research and compare two EAs with the self-adaptive mutation method and different survivor selection methods, namely the comma strategy survivor selection and the age-based survivor selection. With that, we aim to gain insight into the effect of the balance between diversity and quality.

The two forces of diversity and quality are crucial for the performance of a high quality EA. Population diversity is accomplished due to the methods of recombination and mutation. It ensures novelty of the population over the time of generations and therefore avoids local optima and premature convergence. Quality is ensured by parent and survivor selection, which decreases diversity. 

In order to balance the population diversity initiated by the self-adaptive mutation method and enhance population quality, we chose the "comma strategy"\cite{series/ncs/EibenS03}. For comparison, we tested the stated EA against a second algorithm in which an age-based survivor selection method was implemented. Therefore, the second algorithm should have more diverse offspring and hence less quality. This experiment aims to answer the research question whether an EA with a comma survivor selection strategy, which displays a more balanced approach between diversity and quality, can be seen as superior as an EA with an age-based replacement strategy, and thus higher diversity and lower quality. We investigate the research question by comparing two EAs with a different survivor selection strategy in the EvoMan environment \cite{defranca2020evoman} using the evolutionary computation framework DEAP \cite{DEAP_JMLR2012}.\par

In the gaming framework EvoMan \cite{defranca2020evoman}, an agent fights against enemies equipped with different weapons and movements. Movements of the agent include moving left or right, shoot, jumping, and interrupt the jump. The agent and enemies have 100 energy points, which decrease once they are hit by each other's projectiles. The game stops if one of the energy levels turn zero. For our experiments, the two algorithms are tested against two sets of enemies.