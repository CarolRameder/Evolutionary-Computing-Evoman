%\documentclass[sigconf, authordraft]{acmart}
\documentclass[sigconf]{acmart}

\usepackage{newclude}
\usepackage{indentfirst}
\usepackage{subfig}
\usepackage{hyperref}

\usepackage{biblatex}
\addbibresource{references.bib}
% Copyright
%\setcopyright{none}
%\setcopyright{acmcopyright}
%\setcopyright{acmlicensed}
% \setcopyright{rightsretained}
%\setcopyright{usgov}
%\setcopyright{usgovmixed}
%\setcopyright{cagov}
%\setcopyright{cagovmixed}


\begin{document}
\title{\vspace{-3em}Evolutionary Computing}   
\subtitle{Task 2: Generalist Agent \\ Group 23}


%%% The submitted version for review should be ANONYMOUS
\author{Kleio Fragkedaki}
\affiliation{%
   Student number: 2729842
}
\author{Matilda Knierim}
\affiliation{%
   Student number: 2700374
}

\author{Carol Rameder}
\affiliation{%
   Student number: 2747982
}

\author{Robin Stöhr}
\affiliation{%
   Student number: 2750340
}

% \begin{abstract}
% This research paper intends to investigate the difference between an evolutionary algorithm (EA) using a self-adaptive mutation and an EA using a simple Gaussian mutation. Based on the advantages a dynamic adjustment of high or low diversity changes for individual chromosomes inherits, this research investigates whether an EA with a self-adaptive mutation can be seen as superior in comparison to an EA with a standard Gaussian mutation. To test the different methods, an agent playing the game MegaMan of the evoman framework was created. The implementation of the evolutionary algorithm was conducted in the DEAP framework. Amongst other things, this research aimed to bring insight into the differentiation of importance of \textcolor{green}{diversity} of EAs and in a context of hybridization of EAs and Neural Networks.
% \end{abstract}

%
% The code below should be generated by the tool at
% http://dl.acm.org/ccs.cfm
% Please copy and paste the code instead of the example below. 
%


% \keywords{genetic algorithm, evolutionary computing, corssover, mutation}
\maketitle
\include*{Chapters/Chapter1_Introduction}
\include*{Chapters/Chapter2_Related_work}
\include*{Chapters/Chapter3_Algorithm_Description}
\include*{Chapters/Chapter4_Experimental_Setup}
\include*{Chapters/Chapter5_Experimental_Results}
\vspace{-3em}
\include*{Chapters/Chapter6_Analysis_and_Discussion}
\include*{Chapters/Chapter7_Conclusions}
\include*{Chapters/Chapter8_GroupWork}
%\include*{Chapters/Chapter9_Bibliography}
\clearpage
%\bibliographystyle{ACM-Reference-Format}
%\bibliographystyle{ieeetr}
%\bibliography{references}
\printbibliography


\end{document}